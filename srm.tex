\documentclass[11pt]{article}
\usepackage[letterpaper, left=.8in, top=0.9in, right=.8in, bottom=0.70in,nohead,includefoot, verbose, ignoremp]{geometry}
%\usepackage{charter} %choose default font ... your choice here % {mathptmx} % mathrsfs} % mathptmx} %mathpple} %mathpazo}
\usepackage[round]{natbib}
\usepackage{enumerate} % for different labels in numbered lists 
%\usepackage{xy}\xyoption{all} \xyoption{poly} \xyoption{knot}  
\usepackage{latexsym,amssymb,amsmath,amsfonts,graphicx,color,amsthm,enumerate,natbib,mathtools, bm} %,fancyvrb,movie15
\usepackage{dsfont}
\usepackage[pdftex,pagebackref=true]{hyperref}
\usepackage[svgnames,dvipsnames,x11names]{xcolor}
\hypersetup{
colorlinks,%
linkcolor=RoyalBlue2,  % colour of links to eqns, tables, sections, etc
urlcolor=Sienna4,   % colour of unboxed URLs
citecolor=RoyalBlue2  % colour of citations linked in text
}
\pagestyle{empty} % no page number on front page
\usepackage{todonotes}
%\usepackage{mathrsfs} % For \mathscr function

%\renewcommand{\includegraphics}{}  % use this to suppress inclusion of figs for proofing

% custom definitions ...
\def\eq#1{equation (\ref{#1})}
\def\pdf{p.d.f.\ } \def\cdf{c.d.f.\ }
\def\pdfs{p.d.f.s} \def\cdfs{c.d.f.s}
\def\mgf{m.g.f.\ } \def\mgfs{m.g.f.s\ }
\def\ci{\perp\!\!\!\perp}                        % conditional independence symbol
\def\beginmat{ \left( \begin{array} }
\def\endmat{ \end{array} \right) }
\def\diag{{\rm diag}}
\def\log{{\rm log}}
\def\tr{{\rm tr}}
\def\etr{{\rm etr}}
\def\Ei{{\rm Ei}}
\def\cD{\mathcal{D}}
\def\cS{\mathcal{S}}
\def\cM{\mathcal{M}}
%\newtheorem{thm}{Theorem}
\newcommand{\Reals}{\mathbb R}
\newcommand{\Beta}{\mathrm{Beta}}
\newcommand{\KL}{\mathrm{D}_{KL}}
\newtheorem{thm}{Theorem}[section]
\newtheorem{lemma}{Lemma}[section]
\newcommand{\df}{\vcentcolon=}
\newcommand{\ex}{{\mathbb E}}
\newcommand{\pfrac}[2]{\left(\frac{#1}{#2}\right)}

\DeclareBoldMathCommand{\balpha}{\alpha}
\DeclareBoldMathCommand{\bbeta}{\beta}
%

%%My Definitions
\def\qed{\hfill $\square$}
\newcommand{\indep}{\mathop{\perp\!\!\!\!\perp}}

\def\ts{\tilde{s}}

%% Document starts here ...
%%
\begin{document}
\vspace{-1in}
\title{SSRM: A Sequential Hypothesis Test for Compliance in Simply Randomized Experimental Designs}
\author{\Large Michael Lindon \\ Optimizely \and Alan Malek \\ Optimizely?}
\maketitle 
%\centerline{{\color{RoyalBlue2}{Due: someday.}}}\bigskip
%\thispagestyle{empty}
\begin{abstract}
  The randomized \todo{I like your spelling} assignment of experimental units among treatment groups is a key requirement in experimental designs seeking to estimate causal estimands \todo{awkward w.r.t. estimate two words ago} such as the average treatment effect. Failure of the randomization mechanism can introduce a selection bias, rendering causal conclusions biased\todo{biased has a more specific meaning; consider removing} and invalid. This is often observed in online controlled experiments (OCEs), whereby the assignment mechanism is performed in an automated manner and data is collected through complex processing pipelines. A very strong indicator of unexpected behaviour, either in the randomization or data processing, is when the total unit counts in each treatment group differs significantly from what would be expected under the intended randomized design. Such errors are frequently caused by bugs in code or incorrect logic in data processing. In order to validate compliance \todo{is this term well defined?} of the experimental design, we introduce a novel \todo{is this test really novel, or its the application to the online setting novel?} sequential hypothesis test based on Bayesian conjugate multinomial-Dirichlet families. While Bayesian in construction, this test controls frequentist Type-I error under both optional stopping and continuation and prevents users from potentially inflating false positive probabilities through continuous monitoring. This work studies the martingale properties of the posterior odds under the null hypothesis and uses maximal inequalities to bound the frequentist Type-I probability of a stopping rule used to reject the null. \todo{this sentence is probably too much detail for an abstract} This test further posesses the desirable property of having asymptotic power 1, rejecting the null almost surely when the alternative is true. Always valid confidence sequences \todo{this term is also not used enough to be meaningful to our audience; we should just intuitively describe it instead} are derived from this test for estimation purposes and their conenction to the Bayesian support interval is discussed. This test \todo{repetition} provides practitioners a tool to identify problems in the execution of a simply randomized design while data is still being collected and improves upon other procedures which can only be performed after data is collected \todo{you mean after all the data?}. Through simulation studies, we demonstrate that the proposed sequential test rejects the null quickly, preventing subsequent experimental units from entering a faulty experiment. Bayesian and Frequentist approaches finally are bridged via a conditional frequentist testing perspective \todo{this sentence sounds like we solved the bayesian/frequentist debate once and for all}.
\end{abstract}


\section{Introduction}
Experimental design has recently received considerable attention from experimentation teams in the technology sector.
Modern experiments often take the form of online controlled experiments (OCE's), whereby product, marketing and developer teams expose a large number of experimental units (usually online users or visitors) to variations in their product in order to make data driven decisions.
The emphasis is predominantly on the \textit{causal effect} of a new version of the product, relative to the current version, permitting the interpretation that releasing the new version would cause the estimated shift in the outcomes being studied. \todo{I don't think causal effect is well defined until we have a control and a treatment.}
This allows experimentation teams to recommend improvements to the product which would cause positive changes.
In what follows, any new feature or version will be referred to as a treatment, and the existing or current version referred to as a\todo{the?} control.
One frequently made error in to attribute the difference in observed outcomes between treatment and control groups to the causal effect of a treatment when it is, in fact, caused by an incorrectly executed experiment. Such faulty experiment could arise under a selection bias in treatment assignment or when data is missing is some systematic way.

In contrast to \textit{completely randomized experiments}, in which the total number of units to assign to every treatment group is determined apriori as part of the design and the randomization is performed under these constraints\todo{consider removing ``the randomization is performed under these constraints''}, OCEs are typically \textit{simply randomized designs}, whereby new units entering the experiment are assigned to treatment groups independently of other units and according to probabilities determined apriori as part of the design \todo{doesn't simply randomized implies that this probability is also constant?}. The assignment of a unit can therefore be regarded as a multinomial random variable of size 1 with a probability vector $\theta_0$ determined as part of the design. It is a concern, therefore, when the total number of units (counts) in each treatment group is significantly different from what would be expected under multinomial assignment with probability $\theta_0$. \todo{before this sentence, I would introduce, in mathematical terms, the model; i.e. the allocation of unit i, x_i, are sampled i.i.d. from a multinomial with a fixed parameter}  In the online experimentation community, this observation is colloquially referred to as a sample ratio mismatch (SRM) and is highly indicative of an incorrectly executed experiment. The most common causes of SRMs are invalid \todo{what does invalid mean?} treatment assignment usually caused by software errors, and missing data issues usually caused by loss of telemetry data or by invalid data processing logic. As the engineering effort to execute OCEs is considerable, such errors are much more likely than one would expect, with \cite{fabijan} reporting at least 6\% of all experiments performed throughout the course of a year contained implementation bugs revealed by SRMs. As a consequence of just how easy it is to make such engineering mistakes when implementing OCEs, and the consequences of falsely interpreting a faulty experiment, it has become a common practice to validate the implementation by testing
\begin{align}
  \label{eq:srm-hypothesis-test}
  H_0:&\hspace{0.2cm} x_1, x_2, \dots \sim \text{Multinomial}(1,\theta_0),\\
  \text{against}\hspace{0.5cm} H_1:&\hspace{0.2cm} x_1, x_2, \dots \sim \text{Multinomial}(1, \theta) \hspace{0.5cm} \theta \neq \theta_0,
\end{align}
\todo{consider writing this equation to emphasize the sample-size dependence}
using a simple chi-squared test once data collection has been terminated. The fundamental problem with this approach is that one must wait until the end of the experiment in order to perform validation. If the null hypothesis is rejected, the experiment is typically scrapped, meaning in decision theoretic terms that the loss function is 0-1 and independent of the magnitude of the deviation from the null hypothesis \todo{unless we are proposing a decision theoretic analysis, phrasing the problem in decision theoretic terms in not necessary}. The desire to learn of an implementation error as early as possible has led many practitioners to incorrectly \textit{continuously monitor} \todo{define please} their experiments - performing numerous chi-squared tests while the data is still being collected without applying appropriate multiplicity corrections, dramatically inflating their frequentist type-I error. \todo{having n-dependent notation above would make a precise statement here easier}.\todo{try to motivate why we can model the data generation process as i.i.d. mulitnomial}

This paper presents a novel solution to validating a simply randomized experimental design by providing a \textit{sequential} analogue of \ref{eq:srm-hypothesis-test}, in that one can continuously monitor an experiment, with optional stopping and continuation, while still maintaining a frequentist type-I error probability below a nominal level. This safely permits the online testing of hypothesis \ref{eq:srm-hypothesis-test} after every single observation, without inflating false positive probabilities, with the obvious advantage of being able to safely reject the null and discover an implementation error early in the beginning of an experiment - preventing experimental units being wasted on a faulty experiment.

The paper is outlined as follows. Section \ref{sec:causes_of_srms} motivates our contribution by reviewing example causes and consequences of implementation errors in the literature that were revealed by SRMs. Section \ref{sec:causal_models} reviews concepts of causal inference to establish why improper randomization, consequently SRMs, is a cause of concern when the purpose of the experiment is to estimate causal effects. It also reviews some concepts of missing data models which, depending on the goal of the experiment, demonstrate that an SRM does not always invalidate the experiment. Section \ref{sec:srm_testing} outliness the Bayesian construction of this test through conjugate multinomial-Dirichlet models. Section \ref{sec:theory} provide bound on the frequentist error of this test as well as instance-specific upper-bounds on the time-to-rejection. These upper bound are stated in terms of the KL-divergence between $\theta_0$ and the actual generating distribution of the samples.

\section{Causes of Sample Ratio Mismatches}
\label{sec:causes_of_srms}
As the architecture required to implement OCEs often becomes considerably complex there is generous opportunity for code bugs to affect data collection, logical errors to invalidate the data processing rules, and network issues to cause the loss of telemetry data.
\cite{fabijan} describes an experiment in which the number of rotating cards on a carousel is increased between control and treatment.
It was hypothesized that providing more content would result in increased user interaction with the carousel and drive a positive increase in metrics measuring user engagement. Such a hypothesis was reasonable based on prior experience with related experiments, yet the surprising outcome of this experiment showed that the treatment effect was significantly negative - the number of clicks per user appeared to be much less \todo{fewer?} for the treatment than the control. To take the results of this experiment at face value, one would incorrectly conclude this change significantly harmed user engagement. The authors describe that this surprising result was explained by invalid logic in the data processing stage. As the purpose of the experiment is to study the effect on \textit{users}, it is necessary to remove non-human interaction with the carousel from the analysis, such as bots, spiders, web-scrapers etc. Due to the differences between treatment and control, the most engaged users in the treatment were accidentally algorithmically classified as bots and removed from the analysis, effectively removing observations which best supported the alternative hypothesis. \todo{I think this paragraph is mostly unecessary. I think that simply stating that ``this paper has shown it is a significant problem,'' as you did in the introduction, is sufficient.}

\cite{zhao} note that instrumentation and logging of telemetry data is imperfect and results in a gap between the actual vs logged behaviour of users. This is particularly problematic if the rate of information loss is unbalanced between treatment and control, introducing a confounding factor which can bias estimation of the average treatment effect. The authors discuss an example in which the treatment variation of a web page took longer to load than the control. Users exposed to the treatment with older browsers or slower connections were more likely to abandon the page, not providing enough time for their data to be logged. This resulted in fewer experimental units being recorded in the treatment group, and users with slower connections or older technologies being underrepresented in the treatment sample.
\todo[inline]{I would take this whole section, condense it into a sencence or two, and put it in a related works section.}.\todo[inline]{Also, we need a related works section. I can send you some references for the always-valid literature if you need them.}


\section{The Role of Randomization and Missing Data Models}
\label{sec:causal_models}
Simple random sampling is one of the simplest assignment mechanisms used to justify causality \todo{not sure what ``justify causality'' means}. Under a potential outcomes framework of causality each experimental unit $i$ has two potential outcomes associated with it, $Y_i(0)$ and $Y_i(1)$, corresponding to the observed outcomes if this unit were to be assigned to the control or treatment, respectively. Furthermore, consider a fully specified parametric model for the potential outcomes and any pre-treatment covariates
\begin{align*}
  Y(0),Y(1)|X,\theta &\sim p (Y(0),Y(1)|X,\theta),\\
    X|\theta &\sim p (X|\theta).
\end{align*}\todo{define $X$. Why do we even need covariates, though?}
Inference then pertains to questions about $\theta$. For instance, a common goal of an experiment is to estimate causal estimands such as the \textit{average treatment effect} (ATE), or the ATE within subpopulations identified by pre-treatment covariates, defined as
\begin{align*}
  \tau(x,\theta) &= \mathbb{E}[Y_i(1)-Y_i(0)|X_i=x, \theta],\\
  \tau(\theta) &= \mathbb{E}[Y_i(1)-Y_i(0)|\theta],
\end{align*}
respectively. 
The \textit{fundamental problem of causal inference}, however, is that at most one of these outcomes can be observed, the other remaining counterfactual.
Let $W_i\in \lbrace 0,1 \rbrace$ be a treatment indicator resulting from a probabilistic assignment mechanism denoting whether unit $i$ is assigned to the control or treatment group. The observed and missing potential outcomes can then be expressed as
\begin{align*}
  Y_{i,obs} &= W_iY_i(1)+(1-W_i)Y_i(0),\\
  Y_{i,mis} &= (1-W_i)Y_i(1)+W_iY_i(0),\\
\end{align*}
hence $(Y_{mis}, Y_{obs},W) = g(Y(0),Y(1), W)$.
The posterior based on the observed outcomes can be obtained by marginalizing over the unobserved outcomes
\begin{equation}
  \label{eq:full_likelihood}
  p(\theta|Y_{obs}, X, W) \propto \int p(W|Y_{obs},Y_{mis},X) p(Y_{obs},Y_{mis}|X, \theta) p(X,\theta) dY_{mis},
\end{equation}
which is crucially not equal to the posterior which would be obtained from using the likelihood of the observed outcomes alone. This can be obtained with a further assumption of \textit{ignoreability}, that the assignment mechanism is independent of the potential outcomes given the pre-treatment covariates i.e. $W \perp Y(0), Y(1) | X$. The assumption of ignoreability together with a probabilistic assignment simplifiess \eqref{eq:full_likelihood} to
\begin{align*}
  p(\theta|Y_{obs}, X, W) &\propto \int p(W|X) p(Y_{obs},Y_{mis}|X, \theta) p(X,\theta) dY_{mis},\\
  &\propto p(Y_{obs}|X,\theta)p(X,\theta).
\end{align*}
Simple random sampling, therefore, allows the data collection mechanism to be completely ignored and inference on causal estimands can be performed using the posterior distribution obtained from conditioning on the observed outcomes alone ``as usual''. Had the assignment mechanism not be ignoreable, it would be incorrect to work with the observed likelihood $p(Y_{obs}|X,\theta)$ instead of the marginalized complete likelihood $p(W, Y_{obs}|X,\theta)$.

%When a sample ratio mismatch is observed in an experiment and the difference is believed to be caused due to missing data, practitioners are neither willing to assume the missingness process is ignorable, nor are willing to model the missingness. The reluctance to model the missingness process is understandable because there are simply too many mechanisms by which data could be lost, which will be the focus of section \ref{causes_of_srms}, too many experiments being performed and OCEs are relatively cheap. It is far more common to test for an SRM and simply repeat the experiment after correcting the bug. Testing for SRMs is the focus of section \ref{srm_testing} 

\todo[inline]{I have a question for you: why use the potential outcomes framework? We only need some argument that SRM will probably indicate a faulty experiment design, but there are probably easier ways to justify this claim than with counterfactuals. Maybe there isn't, but potential outcomes is pretty unintuitive for many people.}

\section{Sequential Testing of a Simply Randomized Design}
\label{sec:srm_testing}
Let $x_1, x_2 ,\dots $ be a sequence of i.i.d. $\text{Multinomial}(1,\theta)$ random variables and consider the task of developing a sequential test for the simple hypothesis $\theta=\theta_0$. To do this, we consider the sequential properties of the Bayes factor resulting from the following model. Denote the null model as $M_0$ 
\begin{align}
    x_1, x_2, \dots &| M_0 \sim \text{Multinomial}(1,\theta_0),
\end{align}
and the alternative model as $M_1$
\begin{align}
  \theta &| M_1 \sim \text{Dirichlet}(\alpha),\\
  x_1, x_2, \dots &| \theta, M_1 \sim \text{Multinomial}(1,\theta).\notag
\end{align}
The Posterior odds in favour of $M_1$ to $M_0$ after observing $x_{1:t}$ is defined as
\begin{align}
  \label{eq:general_posterior_odds}
  \frac{p(M_1|x_{1:t})}{p(M_0|x_{1:t})}  &= \frac{\int p(x_{1:t}|\theta,M_1)p(\theta,M_1)d\theta}{p(x_{1:t}|M_0)}\frac{P(M_1)}{P(M_0)},\\
                      &=\frac{p(x_{1:t}|M_1)}{p(x_{1:t}|M_0)}\frac{p(M_1)}{p(M_0)},\\
                      &=\frac{\prod_{i=1}^{t}p(x_i|x_{1:i-1}|M_1)}{\prod_{i=1}^{t}p(x_i|x_{1:i-1}|M_0)}\frac{p(M_1)}{p(M_0)},\\
                      &=\frac{p(x_t|x_{1:t-1},M_1)}{p(x_t|x_{1:t-1},M_0)} \frac{p(M_1|x_{1:t-1})}{p(M_0|x_{1:t-1})},\\
    &=\frac{\int p(x_t|\theta,x_{1:t-1},M_1)p(\theta|x_{1:t-1},M_1)d\theta}{p(x_t|x_{1:t-1},M_0)}  \frac{p(M_1|x_{1:t-1})}{p(M_0|x_{1:t-1})} ,
\end{align}
where the last expression really stressess the recursive definition of the Bayes factor in terms of products of posterior predictive densities. The posterior distribution of $\theta| x_{1:t}, M_1 \sim \text{Dirichlet}(\alpha_t)$ where $\alpha_t = \alpha_{t-1}+x_t$ with $\alpha_0$ the initial prior parameter choice. The posterior predictive densities are easily computed as
\begin{equation}
  \label{eq:posterior_predictive_m1}
   p(x_t|x_{1:t-1},M_1) = \frac{ \Gamma(\sum_i x_{t,i}+ 1)}{\prod_i \Gamma(x_{t,i} + 1)} \frac{\Gamma(\sum_i \alpha_{t-1,i})}{\prod_i \Gamma(\alpha_{t-1,i})} \frac{\prod_i \Gamma(\alpha_{t-1,i} + x_{t,i})}{\Gamma(\sum_i \alpha_{t-1,i} + x_{t,i})},
\end{equation}
and
\begin{equation}
  \label{eq:posterior_predictive_m2}
   p(x_t|x_{1:t-1},M_0) = \frac{ \Gamma(\sum_i x_{t,i} + 1)}{\prod_i \Gamma(x_{t,i} + 1)} \prod p_i^{x_{t,i}}.
 \end{equation}
 It will be useful later on to introduce the following notation for the posterior odds at time $t$ as $O_t(\theta_0)$, which explicitly states the value of $\theta$ under the null hypothesis. The recursive definition of the posterior odds can then be expressed as
\begin{align}
  O_{t}(\theta_0) &= \frac{\Gamma(\sum_i \alpha_{t-1,i})}{\Gamma(\sum_i \alpha_{t-1,i} +  x_{t,i})} \frac{\prod_i \Gamma(\alpha_{t-1,i} + x_{t,i})}{\prod_i \Gamma(\alpha_{t-1,i})} \frac{1}{\prod_i \theta_{0,i}^{x_{t,i}}}  O_{t-1}(\theta_0),\\
\end{align}
with
\begin{align}
  \label{eq:alpha_update}
  \alpha_{t}&= \alpha_{t-1}+x_t.
\end{align}
and initial value
\begin{align}
  \label{eq:bayes_factor_seed}
O_0(\theta_0) = \frac{p(M_1)}{p(M_0)}.
\end{align}
Recall that for integer values $\Gamma(n)=(n-1)!$. In addition, $x_{t,i}$ is zero for all but one $i$. Let $j$ denote the index such that $x_{t,j}=1$. If $\alpha_0$ is integer valued, therefore, the recursive definition simplifies substantially to
\begin{align}
  \label{eq:simplified_bayes_factor}
  O_{t}(\theta_0) &= \frac{\alpha_{t-1,j}}{\sum_i \alpha_{t-1,i}} \frac{1}{\theta_{0,j}} O_{t-1}(\theta_0),\\
  &=\frac{E[\theta_j|x_{1:t-1}]}{\theta_{0,j}}  O_{t-1}(\theta_0),
\end{align}
where the last line follows from the mean of the Dirichlet posterior distribution. This multiplicative update has some intuitive appeal - it is the expected probability, based on our current Bayesian belief, divided by the null probability of the event that occured. We now consider properties of the sequence of posterior odds $\lbrace O_{i}(\theta_0) \rbrace_{i=1}^\infty $

\begin{thm}
  $O_i(\theta_0)$ as defined in equation \eqref{eq:general_posterior_odds} is a nonnegative margingale under $M_0$.
  \label{thm:posterior_odds_martingale}
    \end{thm}
  \begin{proof}
  \begin{align}
    E_{M_0}[O_{t+1}(\theta_0)|F_{t}]  &= \int \frac{p(x_{t+1}|x_{1:t},M_1)}{p(x_{t+1}|x_{1:t},M_0)} O_{t}(\theta_0) p(x_{t+1}|x_{1:t},M_0) d_{x_{t+1}}\\
    &=  O_{t}(\theta_0) \int p(x_{t+1}|x_{1:t},M_1) d_{x_{t+1}}\\
    &=  O_{t}(\theta_0).
  \end{align}
\end{proof}
In developing a test we might consider rejecting the null hypothesis as soon as the posterior odds in favour of the alternative exceed some tolerance. The following theorem establishes the probability of this event ever happening under the null, and hence the frequentist Type I error probability.
\begin{thm}
  \label{thm:type_1_error}
Let $x_1, x_2,\dots $ be a sequence of $\text{Multinomial}(1,\theta)$ random variables and consider the sequence $O_i(\theta_0)$ under the assumptions of theorem \ref{thm:posterior_odds_marginale}. Let $u >0$, then
\begin{equation}
  P_{\theta = \theta_0}\left( \sup_{i\in N} O_i(\theta_0) > 1/u \right) \leq u.
\end{equation}
\end{thm}
\begin{proof}
It follows from thorem \ref{thm:posterior_odds_martingale} that $O_i(\theta_0)$ is a nonnegative supermartingale under $M_0$ by definition. The result follows from an application of Ville's maximual inequality for nonnegative supermartingales. An alternative and self contained proof avoiding martingale theory is provided in \cite{robbins}. Yet another proof is provided by Wald's sequential probability ratio test for composite alternatives \cite{wald}.
\end{proof}
This suggests that if one continues to collect data while $O_i(\theta_0) \leq 1/u$ and rejects the null hypothesis as soon as $O_i(\theta_0) > 1/u$, then the probability of this happening under the null, resulting in a false positive, is at most $u$. Theorem \ref{thm:type_1_error} then suggests that this test is able to control the frequentist Type I error probability at a nominal level $\alpha$. This property is, for example, also shared by a trivial test which never rejects the null. In order for this sequential test to be compelling, it must also be able to reject the null with high probability when the null is not true.

\begin{thm} Let $\rho_t(\theta_0) = 1/(1+O_t(\theta_0))$ i.e. the posterior probability at time $t$ that the null hypothesis is correct. Under $M_0$ $E_{M_0}[\rho_{1+1}(\theta_0)|F_t] \geq \rho_{t}(\theta_0)$ i.e. the expected value of the posterior probability in favour of the null hypothesis is greater than or equal to the present value.
\end{thm}

\begin{proof}
  \begin{align}
      E_{M_0}\left[\frac{1-\rho_{t+1}(\theta_0)}{\rho_{t+1}(\theta_0)} | F_t \right] = \frac{1-\rho_{t}(\theta_0)}{\rho_{t}(\theta_0)},
  \end{align}
  by theorem \ref{thm:posterior_odds_martingale}. Yet $f(x)=(1-x)/x$ is a convex function, and so by Jensens inequality
    \begin{align}
      E_{M_0}\left[\frac{1-\rho_{t+1}(\theta_0)}{\rho_{t+1}(\theta_0)} | F_t \right] \geq \frac{1-E_{M_0}[\rho_{t+1}(\theta_0)|F_t]}{E_{M_0}[\rho_{t+1}(\theta_0)|F_t]}.
    \end{align}
    Combining these two conditions gives
    \begin{align}
       \frac{1-\rho_{t}(\theta_0)}{\rho_{t}(\theta_0)} \geq \frac{1-E_{M_0}[\rho_{t+1}(\theta_0)|F_t]}{E_{M_0}[\rho_{t+1}(\theta_0)|F_t]},
    \end{align}
    which implies $E_{M_0}[\rho_{t+1}(\theta_0)|F_t] \geq \rho_{t}(\theta_0)$ 
\end{proof}

I think the next theorem is necessary to demonstrate that this isn't a trivial test (never rejecting the null).
  \begin{thm}
  Asymptotically power 1. For any $p>0$
 \begin{equation} 
  P_{\theta \neq \theta_0}\left(B(x_{1:t};\theta_0) < 1/p \, \forall t \geq 1\right) = 0.
  \end{equation}
\end{thm}
\begin{proof}
  This is related to the consistency of Bayes factors. Consistency in this sense is that $B\rightarrow_p 0$ when $\theta=\theta_0$ and $B\rightarrow_p \infty$ when $\theta\neq \theta_0$ but contained in the support of $M_1$. There's a very in depth review paper by \cite{chib_bf_consistency}, but I think it builds more machinery than we need. This is a simple nested model, and it might be cleaner to use the construction in \cite{fractional_bf}, section 1.3 - asymptotics of Bayes factors. I think the proof here can be used to justify asymptotic power 1
\end{proof}


\begin{thm}
  For $p>0$ let $I_t = \lbrace \theta \in \mathcal{S}^d : B(x_{1:t}|\theta) \leq 1/p \rbrace $, then
  \begin{equation}
    \label{eq:always_valid_ci}
    P_{\theta}\left(\theta \in \bigcap_{t=1}^{\infty} I_t\right) > 1-p
  \end{equation}
\end{thm}
\begin{proof}
  This follows almost automatically from \ref{always_valid_p_value}. Yet there is more interesting discussion here. The Bayes factor can be expressed interstingly int erms of the Savage-Dickey density (\cite{dickey}) ratio i.e.
  \begin{equation}
    B(x_{1:t}|\theta_0) = \frac{p(\theta_0| M_1)}{p(\theta_0|x_{1:t},M_1)},
  \end{equation}
\end{proof}
This the ratio of posterior and prior probability densities under $M_1$ evaluated at the null parameter value. This means that the intervals $I_t$ can be expressed as
$I_t = \lbrace \theta \in \mathcal{S}^d : p(\theta_0| M_1)\leq p(\theta_0|x_{1:t}, M_1)/p \rbrace $, which is identically equal to the \textit{support interval} proposed in \cite{support_interval} (the authors have no idea this interval has frequentist coverage properties).


\section{Indifference Region}

\begin{equation}
  A_n := \left\{ S_n : \frac{\Beta(\alpha_1 + S_n, \alpha_2 + n - S_n)}{\Beta(\alpha_1, \alpha_2)} \frac{1}{p_0^{S_n}(1-p_0)^{n-S_n}} \leq \frac{1}{\alpha} \right\}
\end{equation}

We want to show that this eventually degenerates to $\lbrace p_0 \rbrace$
\begin{align}
  \text{Beta}(x, y) &= \frac{\Gamma(x)\Gamma(y)}{\Gamma(x+y)}\\
  &=\frac{(x-1)!(y-1)!}{(x+y-1)!}
\end{align}
Consider Stirlings approximation
\begin{equation}
  \label{eq:stirling}
  x! = \sqrt{2\pi x} \left( \frac{x}{e} \right)^x   e^{\lambda_x},
\end{equation}
where $\lambda_x$ is $o(1)$ as $x\rightarrow \infty$. Let's just take $\alpha_1=\alpha_2=1$ for now, corresponding to a uniform prior
\begin{align*}
  \log \text{Beta}(S + 1, n - S + 1) =& \log S! + \log (n-S)! - \log n!-\log (n+1)\\
                         =&  \log \sqrt{2\pi S} + S \log S - S \log e  + \lambda_{S}\\
                         &+ \log \sqrt{2\pi (n-S)} +(n-S) \log (n-S) - (n-S) \log e  + \lambda_{n-S}\\
  &- \log \sqrt{2\pi n} - n\log n + n \log e  - \lambda_{n}\\
  &- \log (n+1)\\
  =&  \log \sqrt{2\pi\frac{ S(n-S)}{n}} + n H(S/n) - \log(n+1) + o(1)
\end{align*}.
The log posterior odds can then be approximated as
\begin{align*}
  \log O_n(\theta_0) =&  -S \log \theta_0 - (n-S)\log (1-\theta_0) + \log \sqrt{2\pi\frac{ S(n-S)}{n}} + n H(S/n) - \log(n+1) + o(1)\\
  =&  S\log(\frac{S/n}{\theta_0})  + (n-s) \log(\frac{1-S/n}{1-\theta_0})+ \log \sqrt{2\pi\frac{ S(n-S)}{n}} + - \log(n+1) + o(1)\\
     =&  n KL(\hat{\theta}|\theta_0) +  \log \sqrt{2\pi\frac{ S(n-S)}{n}} - \log(n+1) + o(1)\\
\end{align*}
as $n \rightarrow \infty$, where $\hat{\theta}=S/n$. Letting the test statistic be $\hat{\theta}=S/n$. The region of indifference can then be described as

\begin{equation}
  A_n := \lbrace S_n : KL(S/n|\theta_0)  \leq \frac{\log\left(\frac{1}{\alpha}\right)+\log(n+1) - \log \sqrt{2\pi\frac{ S(n-S)}{n}} + o(1)}{n} \rbrace
\end{equation}


\section{Multivariate Indifference Region}
Consider Stirlings approximation to the log factorial
\begin{equation}
  \label{eq:stirling_log_factorial}
  \log n! = \log\sqrt{2\pi n} + n \log n - n + \lambda_n,
\end{equation}
where $\lambda_n$ is $o(1)$ as $n\rightarrow \infty$. We can use this to approximate the multivariate Beta distribution. First lets consider a uniform distribution over the simplex i.e. $\alpha_i = 1$ for all $i$.
\begin{align*}
  \log \text{Beta}(s+\alpha) =& \sum_i \log \Gamma(s_i + 1) - \log \Gamma(n+d)\\
  =& \sum_i \log s_i! - \log (n+d-1)!\\
  =& \sum_i \left(\log\sqrt{2\pi s_i} + s_i \log s_i - s_i + \lambda_{s_i}\right)  \\
  &- \log\sqrt{2\pi (n+d-1)} - (n+d-1) \log (n+d-1) + (n+d-1) - \lambda_{n+d-1}\\
  =& \log\sqrt{(2\pi)^{d-1}\frac{ \prod_i s_i}{n+d-1}} +  \sum_i\left( s_i \log s_i + \lambda_{s_i}\right)  \\
       &- (n+d-1) \log (n+d-1) + (d-1) - \lambda_{n+d-1}\\
    =& \log\sqrt{(2\pi)^{d-1}\frac{ \prod_i s_i}{n+d-1}} +  \sum_i \left(n\frac{s_i}{n} \log \frac{s_i}n + \lambda_{s_i}\right)\\
       & + n\log n - (n+d-1) \log (n+d-1) + (d-1) - \lambda_{n+d-1}.
\end{align*}

The log posterior odds can then be approximated as
\begin{align*}
  \log O_n(\theta_0) =& \log\sqrt{(2\pi)^{d-1}\frac{ \prod_i s_i}{n+d-1}} +  \sum_i n\frac{s_i}{n} \log \frac{s_i}n -  \sum_i n\frac{s_i}{n} \log \theta_{0i} + \lambda_{s_i},  \\
                      & + n\log n - (n+d-1) \log (n+d-1) + (d-1) - \lambda_n,\\
   =& nKL(\hat{\theta}|\theta_0)+ \log\sqrt{(2\pi)^{d-1}\frac{ \prod_i s_i}{n+d-1}}  + n\log n - (n+d-1) \log (n+d-1) + (d-1)+o(1)\\
\end{align*}
Once again let $\hat{\theta}(x_1,...)$ form our test statistic, then the indifference region can be written as

\begin{equation}
  A_n := \lbrace \hat{\theta}(x) \in  S^d : KL(\hat{\theta}|\theta_0)  \leq \frac{\log\left(\frac{1}{\alpha}\right) - \log\sqrt{(2\pi)^{d-1}\frac{ \prod_i s_i}{n+d-1}}+ (n+d-1) \log (n+d-1) -  n \log n -(d-1)+o(1) }{n} \rbrace
\end{equation}
The right-hand side of the inequality is $o(1)$ i.e.
\begin{equation}
  A_n := \lbrace \hat{\theta}(x) \in  S^d : KL(\hat{\theta}|\theta_0)  \leq g(n) \rbrace,
\end{equation}
where $g(n)$ is $o(1)$.
Let $A_{\infty} = \bigcap_{n=1}^{\infty} A_n$, then $A_{\infty} = \lbrace \theta_0 \rbrace$. First note that $\theta_0 \in A_n \, \forall n$ because $KL(\theta_0|\theta_0) = 0$. For $\theta \neq \theta_0$ we have that $KL(\theta|\theta_0) = \varepsilon >  0$, therefore there exists an $N>0$ s.t. $g(n) < \varepsilon$ for all $n \geq N$ and so $\theta \neq A_n$ for all $n \geq N$.



\section{Bounding the time to significance}
The false positive guarantees the validity of the sequential SRM test, but they are only half of the equation. This section established good performance on the other side; namely, that the SSRM will find true differences quickly. In particular, we provide high probability bounds on the stopping time for data generated from some true alternative distribution $\theta_1$ in terms of some TBD distance measure of $\theta_1 - \theta_0$.

For the general multinomial case, we will use bold greek letters $\balpha = (\alpha_1,\ldots, \alpha_d)$ for vectors (in particular, distinguishing from the scalar $\alpha$); we will also use $\balpha^\bbeta = \prod_{i}^d \alpha_i^{\beta_i}$ to denote elementwise exponentiation, and we will use $|\balpha| = \sum_i\alpha_i$ for the sum. For the vector of counts, we will write $S_n=(S_1^n,\ldots, S_d^n)\in\Reals^d$, which lets of easily define the empirical probability of outcome $i$ as $\hat \theta_i^n \df S_n^i/n$. For any $\balpha,\bbeta\in\Reals^d$, we interpret exponentiation to be elementwise and define 

For convenience, we will also define $\Beta(\bbeta) \df \prod_i \Gamma(\beta_i) / \Gamma(\sum_i\beta_i)$ for any vector $\bbeta$. The test rejects the null hypothesis corresponding to $\theta_0$ when the event
\begin{equation}\label{eq:multinomial.rejection.region}
  \mathcal E_n \df \left\{\frac{\Beta(\balpha + S_n)}
  {\Beta(\balpha)}
  \theta_0 ^{-S_n} \geq \frac{1}{\alpha}\right\}
\end{equation}
occurs.

Our goal is to provide a guarantee of the form: if $S^n\sim\mathrm{multinomial}(\theta_1,n)$, then with probability $\delta>0$, there exists an $N$ such that 
\begin{equation*}
  P \left(\forall n \geq N,\;
      \frac{\Beta(\balpha + S_n)}
  {\Beta(\balpha)}
  \theta_0 ^{-S_n} \geq \frac{1}{\alpha}
  \right) \geq 1-\delta.
\end{equation*}
We would also like to upper bound this $N$ as a function of $\theta_1$ and $\theta_0$.

We will accomplish this goal in two parts. First, we will lower bound 
$
  P \left(
      \frac{\Beta(\balpha + S_n)}
  {\Beta(\balpha)}
  \theta_0 ^{-S_n} \geq \frac{1}{\alpha}
  \right) 
$
as a function of $S_n$. Then, we will construct a confidence sequence on $S_n$ (in terms of $n$ and $\theta_1$) and combine the two.

\subsection{Manipulating the rejection probability}
Our goal is to find some event $\mathcal E_n'\subseteq \mathcal E_n$ that is more amenable to computation; this allows us to obtain a bound on the power since
$P(\forall n \geq N,\; \mathcal E_n) \geq P(\forall n \geq N,\; \mathcal E_n')$.  Specifically, we will prove the following theorem.
\begin{thm}\label{thm:calEprime}
Define, for every $n>0$ and $\alpha, S_n\in(\Reals^+)^d$, the set
\begin{equation}
  \label{eq:decision_boundary}
  \mathcal E_n'\df
  \left\{
    \KL(\hat\theta_n|| \theta)
    \geq
      \frac{|\balpha|^2}{n^2} + \frac{|\balpha|}{n}
    +\frac{1}{n}\log\frac{\Beta(\balpha)}{\alpha}
  \right\}.
\end{equation}
Then $\mathcal E_n' \subseteq \mathcal E_n$ for all $n>0$. 
\end{thm}
This $\mathcal E_n'$ is much more convenient. For example, if $\alpha_i = 1$ for all $i$, then 
\begin{equation*}
  \mathcal E_n'\df
  \left\{
    \KL(\hat\theta_n|| \theta)
    \geq
    \frac{d^2}{n^2}
    +\frac{d}{n}
    +\frac{1}{n}\log\frac{1}{\alpha}
  \right\}.
\end{equation*}

The proof is by explicit construction, and we begin by deriving a lower bound on 
$
 \frac{\Beta(\balpha + S_n)} {\Beta(\balpha)} \theta_0 ^{-S_n}
$.
There are many ways to bound the Beta function; for the two dimensional case, it suffices to use
\begin{equation}\label{eq:Beta.lower.bound}
  \Beta(x,y) \geq \frac{x^{x-1}y^{y-1}}
  {(x+y)^{x+y-1}}\quad\forall x,y>0,
\end{equation}
borrowed from \cite{grenie2015inequalities}. We will need the following multinomial generalization.
\begin{lemma}\label{lem:beta.lower.bound}
  For all $\balpha$ in the positive orthant,
  \begin{equation*}
    \Beta(\balpha) \geq
    \frac{ \prod_i \alpha_i^{\alpha_i-1}}
    {(\sum_i \alpha_i)^{\sum_i \alpha_i-1}}.
  \end{equation*}
\end{lemma}
\begin{proof}
  We mostly follow the derivation of equation (4) in \cite{grenie2015inequalities}. For any completely monotone function $f:(0,\inf)\rightarrow(0,1]$, a result from \cite{kimberling1974probabilistic} yields that
\[
  \frac{f(x+y)}{f(x)f(y)} \geq 1 \; \forall x,y>0,
\]
Applying this identity repeatedly, we see that
\[
  f(|\balpha|)
  \geq
  f(\alpha_1)f\left(\textstyle \sum_{i>1} \alpha_i\right)
  \geq \ldots\geq
  \prod_i f(\alpha_i).
\]
We will apply this identity to the completely monotone function $\exp(-H(x))$, where
\[
  H(x) = x - x\log(x) + \log\Gamma(x+1);
\]
see \cite{grenie2015inequalities} for a justification. This produces
\begin{align*}
  \frac{1}{\Gamma(|\balpha|+1)}e^{-|\balpha|+|\balpha|\log(|\balpha|)}
  &=
  e^{-H\left(|\balpha|\right)}\\
  &\geq
    e^{-\sum_i H\left(\alpha_i\right)}\\
  &=
     e^{\sum_i\left(- \alpha_i + \alpha_i \log(\alpha_i)\right)}\prod_i\frac{1}{\Gamma(\alpha_i+1)}.
\end{align*}
Rearranging yields
\[
  \frac{\prod_i\Gamma(\alpha_i+1)}{\Gamma(|\balpha|+1)}
  \geq
  \frac{\prod_i \alpha_i^{\alpha_i}e^{-\alpha_i}}
  {|\balpha|^{|\balpha|}e^{-|\balpha|}},
\]
which implies the bound
\begin{align*}
  \Beta(\balpha)
  =
  \frac{\prod_i\Gamma(\alpha_i)}{\Gamma(|\balpha|)}
  &=
  \frac{(|\balpha|+1)\prod_i}
  {\prod_i (\alpha_i+1)}
   \frac{\Gamma(\alpha_i+1)}{\Gamma(|\balpha|+1)}\\
  &\geq
  \frac{\prod_i \alpha_i^{\alpha_i}}
  {|\balpha|^{|\balpha|}}
  \frac{(|\balpha|+1)}{\prod_i (\alpha_i+1)}\\
  &\geq
  \frac{\prod_i \alpha_i^{\alpha_i-1}}
  {|\balpha|^{|\balpha|-1}}.
\end{align*}
\end{proof}
We can now finish the proof of Theorem~\ref{thm:calEprime}.
\begin{proof}
Expanding the left hand side of \eqref{eq:multinomial.rejection.region}, and applying Lemma~\ref{lem:beta.lower.bound}, we see
\begin{align*}
    \frac{\Beta(\balpha + S_n)}
  {\Beta(\balpha)} \theta_i^{-S_n}
  &\geq
    \frac{ \prod_i (\alpha_i+S_i^n)^{\alpha_i + S_i^n-1}}
    {(|\balpha|+n)^{|\balpha| + n - 1}}
    \frac{ \theta_i^{-S_n}}{\Beta(\balpha)}\\
  &=
    n^n
    \prod_i
    \left(
    \frac{\frac{\alpha_i}{n} + \hat \theta_i^n}{\theta_i}\right)^{S_i^n}
    \frac{
    \prod_i
    (\alpha_i + S_i^n)^{\alpha_i - 1}
    }
    {(|\balpha|+n)^{|\balpha|+n-1}\Beta(\balpha)}\\
  &\geq
    n^n
    \prod_i
    \left(
    \frac{\hat \theta_i^n}{\theta_i}\right)^{S_i^n}
    \frac{
    \prod_i
    (\alpha_i + S_i^n)^{\alpha_i - 1}
    }
    {(|\balpha|+n)^{|\balpha|+n-1}\Beta(\balpha)}\\
  &=
    n^n
    e^{n \KL(\hat\theta_n|| \theta) }
    \frac{
    \prod_i
    (\alpha_i + S_i^n)^{\alpha_i - 1}
    }
    {(|\balpha|+n)^{|\balpha|+n-1}\Beta(\balpha)}.
\end{align*}
This lets us construct
\begin{align*}
  \mathcal E_n
  &=
  \left\{
    \log \left(
    \Beta(\balpha+S_n) \theta_i^{-S_n}
  \right)
  \geq \log\frac{\Beta(\balpha)}{\alpha}
    \right\}\\
  &\supseteq
    \left\{
        n\log(n)
    +n \KL(\hat\theta_n|| \theta)
    +
    \log\left(\frac{
    \prod_i
    (\alpha_i + S_i^n)^{\alpha_i - 1}
    }
    {(|\balpha|+n)^{|\balpha|+n-1}}\right)
    \geq
    \log\frac{\Beta(\balpha)}{\alpha}
    \right\}\\
  &=
    \left\{
    n \KL(\hat\theta_n|| \theta)
    +
    \sum_i\log\left(
    (\alpha_i + S_i^n)^{\alpha_i - 1}
    \right)
    \geq
    (|\balpha|+n-1)\log(|\balpha|+n)
    -n\log(n)
    +\log\frac{\Beta(\balpha)}{\alpha}
    \right\}\\
    &\supset
    \left\{
    n \KL(\hat\theta_n|| \theta)
    +
    \sum_i\log\left(
    (\alpha_i + S_i^n)^{\alpha_i - 1}
    \right)
    \geq
      \frac{|\balpha|^2}{n} + |\balpha|
    +\log\frac{\Beta(\balpha)}{\alpha}
    \right\},
\end{align*}
where the last line followed because, under the assumption that $|\balpha| \geq 1$
\begin{align*}
    (|\balpha|+n-1)\log(|\balpha|+n)
  -n\log(n)
  &\geq
    (|\balpha|+n-1)\log(|\balpha|+n)
    -(|\balpha|+n-1)\log(n)\\
  &\geq
  (|\balpha|+n-1)
    \log\left(1+\frac{|\balpha|}{n}\right)\\
  &\geq
    (|\balpha|+n-1)\frac{|\balpha|}{n}
    \geq
    \frac{|\balpha|^2}{n} + |\balpha|.
\end{align*}

The last step is to upper-bound
We can upper bound the final remaining summation by
\begin{align*}
  \sum_i(\alpha_i-1)\log
  \left(
  \alpha_i + S_i^n
  \right)
  &\leq
    (|\balpha|-d)\log
    \left(
    |\balpha|+n
    \right),
\end{align*}
showing that
\[
      \left\{
    n \KL(\hat\theta_n|| \theta)
    \geq
      \frac{|\balpha|^2}{n} + |\balpha|
      +\log\frac{\Beta(\balpha)}{\alpha}
      -
          (|\balpha|-d)\log
    \left(
    |\balpha|+n
    \right)
  \right\}
  \subseteq \mathcal E_n,
\]
which implies that $\mathcal E_n' \subseteq \mathcal E_n$ (the bound on the summation is only included for completeness).
\end{proof}

\subsection{Deriving a Confidence Sequence for $S_n$}
The next step in the argument is to derive a confidence sequence for $S_n$ under some alternative hypothesis. In fact, we only need to derive a lower bound on $\KL (\hat\theta_n||\theta_0)$ in order to apply the results of the previous section.

In particular, we will prove a confidence sequence on the total variation norm directly. This confidence sequence is novel and may be of independent interest.
\begin{lemma}\label{lem:CS}
  Fix some $\theta$ in the $d$-simplex and assume that $X_t\sim\mathrm{Multinomial}(\theta)$ for $t>0$. Let $\hat\theta_n$ be the empirical frequencies after $n$ samples. For any $\delta>0$,  we have
\begin{equation}\label{eqn:multinomial.CS}
  P\left( \exists n > 0: \Vert \hat\theta_n - \theta\Vert_1
    \geq
  \frac{2}{n}\sqrt{\left(\frac{n}{4} + \rho\right)\log\pfrac{n/4+\rho}{2^{-2d}\delta^2 \rho}} 
  \right)
  \leq \delta.
\end{equation}  
\end{lemma}
\begin{proof}
  We will follow the construction of \cite{weissman2003inequalities}.
  Using the shorthand $\theta'(A) \df P(Z\in  A)$ where $Z\sim\mathrm{Multinomial}(\theta')$, the total variation and $L_1$ norms are related by
\[
  \Vert \hat\theta_t - \theta\Vert_1
  =
  2 \max_{A \subseteq \{1,\ldots, d\}} \hat\theta_t(A) - \theta(A),
\]
and so we may use a union bound to arrive at
\begin{align*}
  P(\exists n > 0:\; \Vert \hat\theta_n - \theta\Vert_1 \geq \epsilon)
  \leq
  \sum_{A \subset \{1,\ldots, d\}}
  P\left(\exists n > 0:\; \hat\theta_n(A) - \theta(A) \geq \frac{\epsilon}{2}\right).
\end{align*}

Hence, for any $A \subset \{1,\ldots, d\}$, we want to upper bound
\[
 P\left(\exists n > 0:\; \hat\theta_n(A) - \theta(A) \geq \frac{\epsilon}{2}\right).
\]

Fortunately, $n \hat\theta_n(A)$ is the sum of independent Bernoulli random variables $\mathds{1}\{ \Delta S_n \in A\}$ with mean $\theta(A)$ (where $\mathds{1}\{\cdot\}$ is the indicator function). This implies that $n \hat\theta_n(A)$ is the sum of sub-Gaussian random variables with $\sigma^2 = 1/4$, and we may apply any sub-Gaussian mixture boundary. In particular, \cite[Equation~14]{howard2018uniform} implies that
\[
  P\left(\exists n > 0: \hat\theta_n(A) - \theta(A) \geq
    \frac{1}{n}\sqrt{\left(\frac{n}{4} + \rho\right)\log\pfrac{n/4+\rho}{(\delta')^2 \rho}}
  \right)
  \leq
  \delta',
\]
where $\rho$ is mixture variance.
Hence, choosing $\delta' = 2^{-d} \delta$, we have
\begin{align*}
  \lefteqn{
  P\left(\exists n > 0:\; \Vert \hat\theta_n - \theta\Vert_1 \geq
  \frac{2}{n}\sqrt{\left(\frac{n}{4} + \rho\right)\log\pfrac{n/4+\rho}{2^{-2d}\delta^2 \rho}} 
\right)
  }\\
  &\leq
  \sum_{A \subset \{1,\ldots, d\}}
  P\left(\exists n > 0:\; \hat\theta_n(A) - \theta(A)
  \geq
    \frac{1}{n}\sqrt{\left(\frac{n}{4} + \rho\right)\log\pfrac{n/4+\rho}{2^{-2d}\delta^2 \rho}}  
  \right)\\
  &\leq
    \sum_{A \subset \{1,\ldots, d\}}
    \delta 2^{-d}\\
  &\leq
  \delta.
\end{align*}



\end{proof}

We can now combine the two lemmas of this section into a finite-sample power guarantee. We will denote the decision boundary from \eqref{eq:decision_boundary} by
\[\Gamma_n(\balpha)
  \df
  \frac{|\balpha|^2}{n^2} + \frac{|\balpha|}{n}
  +
  \frac{1}{n}\log\frac{\Beta(\balpha)}{\alpha}.
\]
\begin{thm}\label{thm:power_lower_bound}
  Assume that $S_n$ has a multinomial $\tilde\theta$ distribution. For every $\balpha$, $\delta$, $\alpha$, $\tilde\theta$, and $\theta_0$, there exists an $N$ such that
  \begin{equation*}
    P\left(\forall n \geq N: \KL(\hat\theta_n \Vert \theta_0) \geq \Gamma_n(\balpha)\right)
    \geq 1-\delta.
  \end{equation*}
\end{thm}
\begin{proof}
  Define the event
  \[
    \mathcal R_n
    \df
    \left\{
    \KL(\hat\theta_n \Vert \theta_0) \geq \Gamma_n(\balpha)
  \right\}.
\]
By Applying Pinsker's inequality then the triangle inequality, we can show that
\[
  \mathcal R_n \supseteq
  \left\{
    \Vert \hat\theta_n - \theta_0\Vert_1 \geq \sqrt{2\Gamma_n(\balpha)}
  \right\}
  \supseteq
    \left\{
      \Vert \tilde\theta - \theta_0\Vert_1
      -
      \Vert \hat\theta_n - \tilde\theta\Vert_1
      \geq \sqrt{2\Gamma_n(\balpha)}
  \right\}.
\]
Hence, we would like to find an $N$ such that
\[
      P\left(
      \Vert \tilde\theta - \theta_0\Vert_1
      -
      \Vert \hat\theta_n - \tilde\theta\Vert_1
      \geq \sqrt{2\Gamma_n(\balpha)}
  \right)\geq 1-\delta
\]
for all $n\geq N$. By Lemma~\ref{lem:CS}, with probability at least $1-\delta$, the event
\begin{align*}
  \left\{ \forall n > 0:
  \Vert\hat\theta_n- \tilde\theta\Vert_1
  \leq
    \frac{2}{n}\sqrt{\left(\frac{n}{4} + \rho\right)\log\pfrac{n/4+\rho}{2^{-2d}\delta^2 \rho}}\right\}
\end{align*}
occurs.

The existence of an $N$ that satisfied the conditions of the theorem can be show by choosing any $N$ that satisfies
\[
  \Vert \theta_0- \tilde\theta\Vert_{1}
    \geq
        \frac{2}{n}\sqrt{\left(\frac{n}{4} + \rho\right)\log\pfrac{n/4+\rho}{2^{-2d}\delta^2 \rho}}
    + \sqrt{2\Gamma_n(\balpha)}.
  \]
Such an $N$ is well defined because the left hand side is a constant and the right hand side is a decreasing fuction of $N$. In particular, we may simply take $\gamma = 1$; the right most term decays like $1/\sqrt{n}$, so optimizing $\gamma$ will not yield a faster rate. 


Combining the previous few steps, we have
\begin{equation*}
    P\left(\forall n>0 \leq N: \KL(\hat\theta_n \Vert \theta_0) \geq \Gamma_n(\balpha)\right)
    \geq 1-\delta,
  \end{equation*}
  completing the proof.
\end{proof}
\begin{lemma}
  For the $N$ defined by Theorem~\ref{thm:power_lower_bound}, we have
  \begin{equation*}
    N = \Omega\left(
      \frac{d}{\Vert \theta_0- \tilde\theta\Vert_{1}^2}
\log\pfrac{1}{\delta}\log\pfrac{1}{\alpha}
    \right).
  \end{equation*}
\end{lemma}
\begin{proof}
The sample complexity will be the solution to
\[
  N \df
  \min
  \left\{
    n:\;
      \left(1 + \frac{4\rho}{n}\right)
      \log\pfrac{n/4+\rho}{2^{-2d}\delta^2 \rho}
    +
      \frac{2|\balpha|^2}{n^2} + \frac{2|\balpha|}{n}
      +
    \frac{2}{n}\log\frac{\Beta(\balpha)}{\alpha}
    \leq
    \frac{1}{2}\Vert \theta_0- \tilde\theta\Vert_{1}^2
  \right\},
\]
  which has the same order as described in the lemma. To check that this $N$ is sufficient, we will argue that this $N$ ensures that
  \begin{equation}\label{eqn:n.bound}
    \Vert \theta_0- \tilde\theta\Vert_{1}
    \geq
    \frac{2}{n}\sqrt{\left(\frac{n}{4} + \rho\right)
      \log\pfrac{n/4+\rho}{2^{-2d}\delta^2 \rho}}
    + \sqrt{2}\sqrt{
      \frac{|\balpha|^2}{n^2} + \frac{|\balpha|}{n}
      +
      \frac{1}{n}\log\frac{\Beta(\balpha)}{\alpha}
    }.
\end{equation}  

Since convexity of the square root implies  $\sqrt{a}+\sqrt{b}\leq\sqrt{2(a+b)}$, we can show that
\begin{align*}
    \lefteqn{\frac{2}{n}\sqrt{\left(\frac{n}{4} + \rho\right)
      \log\pfrac{n/4+\rho}{2^{-2d}\delta^2 \rho}}
    + \sqrt{2}\sqrt{
      \frac{|\balpha|^2}{n^2} + \frac{|\balpha|}{n}
      +
      \frac{1}{n}\log\frac{\Beta(\balpha)}{\alpha}
  }}\\
  &\leq
    \sqrt{2}
      \sqrt{\frac{1}{n}\left(1 + \frac{4\rho}{n}\right)
      \log\pfrac{n/4+\rho}{2^{-2d}\delta^2 \rho}
    +
      \frac{2|\balpha|^2}{n^2} + \frac{2|\balpha|}{n}
      +
    \frac{2}{n}\log\frac{\Beta(\balpha)}{\alpha}
    },
\end{align*}
which implies that, under $n \geq N$, we have
\begin{align*}
    \Vert \theta_0- \tilde\theta\Vert_{1}
  &\geq
    \sqrt{
    \left(1 + \frac{4\rho}{n}\right)
      \log\pfrac{n/4+\rho}{2^{-2d}\delta^2 \rho}
    +
      \frac{2|\balpha|^2}{n^2} + \frac{2|\balpha|}{n}
      +
    \frac{2}{n}\log\frac{\Beta(\balpha)}{\alpha}
    }\\
    &\geq
      \frac{2}{n}\sqrt{\left(\frac{n}{4} + \rho\right)
      \log\pfrac{n/4+\rho}{2^{-2d}\delta^2 \rho}}
    + \sqrt{2}\sqrt{
      \frac{|\balpha|^2}{n^2} + \frac{|\balpha|}{n}
      +
      \frac{1}{n}\log\frac{\Beta(\balpha)}{\alpha}
      }.
\end{align*}
\end{proof}

\subsection{Extensions}
We would like to choose $\gamma$ in Lemma~\ref{lem:CS} to minimize the upper bound. This means that we would like
For now, assume that we chose $\gamma$ to be the minimizer of the bound
\[
  \gamma^* = \sqrt{ 2\frac{d}{n \Vert \tilde\theta\Vert_2^2}\log\left(\frac{2d}{\delta}\right)},
\]
which guarantees
\begin{align*}
P  \left( \forall n > 0: \Vert\hat\theta - \tilde\theta\Vert_{TV}
  \leq
  \sqrt{ 8\frac{d}{n \Vert \tilde\theta\Vert_2^2}\log\left(\frac{2d}{\delta}\right)}
  \right) \geq 1-\alpha.
\end{align*}
Unfortunately, the $\gamma$ must be chosen ahead of time. We may approximate this by taking a mixture boundary, where we integrate $\gamma$ against some mixture distribution. 

\bibliographystyle{plainnat}
\bibliography{sample}


\appendix

\section{Trying to use a sub-Bernoulli boundary}
The difficulty in trying to use the CGF of the Bernoulli random variables directly is that the typical Chernoff method applied to Bernoullis provide a bound of the form
\[
  \P(\hat\theta_n > (1+\delta)\theta)\leq \exp(-\mu\delta^2/3).
\]
To get an additive bound needed by the rest of the argument, we would need to tune $\lambda$ as a function of $n$. Maybe this goal can be accomplished by the method of mixtures. I've included some partial derivations below in case they come in handy.

Fortunately, $n \hat\theta_n(A)$ is the sum of independent Bernoulli random variables $\mathds{1}\{ \Delta S_n \in A\}$ with mean $\theta(A)$ (where $\mathds{1}\{\cdot\}$ is the indicator function). The cumulant generating function of a Bernoulli random variable is $\psi(\lambda) = \log\left(1 - \theta(1-e^\lambda)\right)$, and so
\[
  P\left(
    \exists n > 0:\;
    \lambda \hat\theta_n(A)
    \geq
    \frac{1}{n}\log\left(\frac{1}{\delta}\right)
    + \log\left(1 - \theta(A)(1-e^\lambda)\right)
  \right)
  \leq \delta.
\]
We can upper bound the cumulant generating function by
\[
  \log\left(1 - \theta(A)(1-e^\lambda)\right)
  \leq
  \theta(A)(e^\lambda-1)
  \leq
  \theta(A)(\lambda+\lambda^2)
\]
where the second inequality holds  on $\lambda \leq 1.79$. Thus, we can show that 
\begin{align*}
  \delta
  &\geq 
    P\left(
    \exists n > 0:\;
    \lambda \hat\theta_n(A)
    \geq
    \frac{1}{n}\log\left(\frac{1}{\delta}\right)
    + \log\left(1 - \theta(A)(1-e^\lambda)\right)
    \right)\\
  &\geq
    P\left(
    \exists n > 0:\;
    \lambda \hat\theta_n(A)
    \geq
    \frac{1}{n}\log\left(\frac{1}{\delta}\right)
    +  \theta(A)(\lambda + \lambda^2)
    \right)\\
  &\geq
    P\left(
    \exists n > 0:\;
    \hat\theta_n(A) - \theta(A)
    \geq
    \frac{1}{\lambda n}\log\left(\frac{1}{\delta}\right)
    +  \theta(A)\lambda
    \right)\\
    &\geq
    P\left(
    \exists n > 0:\;
    \hat\theta_n(A) - \theta(A)
    \geq
    \frac{1}{\lambda n}\log\left(\frac{1}{\delta}\right)
    +  \lambda
    \right)    
\end{align*}
While it might be tempting to optimizintg over $\lambda$ and obtain a $1/\sqrt{n}$ rate, we cannot choose $\lambda$ as a function of $n$ or of the data. A technique such as the method of mixtures \citep{delapena2000moment} would be required. Fortunately, for our case, it suffices to take $\lambda = 1$.

Finally, we set  $\epsilon = (\frac{1}{n}\log(\frac{2^d}{\delta}) + 1)$ and assemble these ingredients together to find
\begin{align*}
  P(\exists n > 0:\; \Vert \hat\theta_n - \theta\Vert_1 \geq \epsilon)
  &\leq
  \sum_{A \subset \{1,\ldots, d\}}
  P\left(\exists n > 0:\; \hat\theta_n(A) - \theta(A) \geq \frac{\epsilon}{2}\right)\\
  &\leq
  \sum_{A \subset \{1,\ldots, d\}}
  P\left(\exists n > 0:\; \hat\theta_n(A) - \theta(A) \geq
    \frac{1}{t}\log\left(\frac{2^d}{\delta}\right) + 1
    \right)\\
  &\leq
  \sum_{A \subset \{1,\ldots, d\}}
    \delta 2^{-d}\\
  & =
    \delta.
\end{align*}

\section{Michael: Using Concentration Inequality}
From the previous sections we note that the \textbf{indifference region}, where
the null hypothesis is not rejected, is
\begin{align}
  I_n = \lbrace x \in S^d : D_{KL}(x||\theta_0) \leq k_n \rbrace,
\end{align}
where $k_n$ is just some constant as a function of $n$. We also note the concentration inequality
\begin{align}
  \mathbb{P}[\hat{\theta}_n \in C_n^{\delta}] \geq 1-\delta,
\end{align}
where
\begin{align}
  C_n^{\delta}=\lbrace x \in S^d : \|x-\theta\|_1 \leq \sqrt{\frac{4\log \frac{2}{\delta}}{n}}\rbrace.
\end{align}
From this we know that with high probability the MLE $\hat{\theta}_n \in C_n^{\delta}$ and, if we can show that $C_n^{\delta} \cap I_n = \emptyset$, then with high probability the null hypothesis is rejected. As a first step lets create a superset of $I_n$ that is defined in terms of the L1 norm instead of the Kullback Leibler divergence. From Pinsker's inequality $2\|x-\theta_0\|_1 \leq D_{KL}(x||\theta_0)$ and so we may define a set
\begin{align}
  J_n = \lbrace x \in S^d : \|x-\theta_0\|_1 \leq \frac{1}{2}k_n \rbrace.
\end{align}
If $x$ satisfies $D_{KL}(x||\theta_0) \leq k_n$, then by Pinsker's inequality $\|x-\theta_0\|_1 \leq \frac{1}{2}k_n$ is satisfied also, and so $I_n \subset J_n$. The next step is to choose a $\delta$ such that $C_n^\delta \cap  J_n = \emptyset$. From an application of the reverse triangle inequality it follows that
\begin{align*}
  C_n^\delta \subset D_n^\delta =& \lbrace x \in S^d : |\|x-\theta_0\|_1 - \|\theta-\theta_0\|_1 | \leq \sqrt{\frac{4\log \frac{2}{\delta}}{n}}\rbrace\\
  &\subset \lbrace x \in S^d : \|\theta-\theta_0\|_1  - \sqrt{\frac{4\log \frac{2}{\delta}}{n}} \leq  \|x-\theta_0\|_1  \rbrace := E_n^{\delta}.
\end{align*}
Let's choose $\delta$ to be a function of $n$, denoted $\delta^n$, satisfying
\begin{align}
  \|\theta-\theta_0\|_1 - \sqrt{\frac{4\log \frac{2}{\delta^n}}{n}}\geq \frac{1}{2} k_n,
\end{align}
then $E_n^{\delta^n} \cap J_n = \emptyset$ which simplies that $C_n^{\delta^n} \cap I_n = \emptyset$. Note that for small $n$ it might not be possible to find such a $\delta^n$, but $k_n$ decreases to zero as a function of $n$. For large enough $n$, therefore, we have that
\begin{align*}
  &\mathbb{P}[\hat{\theta} \in C_n^{\delta^n}] \geq 1-\delta^n\\
  &C_n^{\delta^n} \cap I_n = \emptyset\\
  \Rightarrow &\mathbb{P}[\hat{\theta}_n \in I_n] \leq \delta^n
\end{align*}
TODO: Need to work out the functional form of $\delta^n$ but the sketch of asymptotic power 1 goes like this. Demonstrate that
\begin{align*}
  \sum_{n=1}^{\infty}\mathbb{P}[\theta_n \in I_n] = c \leq \infty,
\end{align*}
then by the Borel Cantelli lemma
\begin{align*}
  \mathbb{P}[\hat{\theta}_n \in I_n i.o.] = 0,
\end{align*}
i.e. the test statistic cannot be in the indifference region infinitly many times.
\end{document}